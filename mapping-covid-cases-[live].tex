% Options for packages loaded elsewhere
\PassOptionsToPackage{unicode}{hyperref}
\PassOptionsToPackage{hyphens}{url}
%
\documentclass[
]{article}
\usepackage{amsmath,amssymb}
\usepackage{lmodern}
\usepackage{ifxetex,ifluatex}
\ifnum 0\ifxetex 1\fi\ifluatex 1\fi=0 % if pdftex
  \usepackage[T1]{fontenc}
  \usepackage[utf8]{inputenc}
  \usepackage{textcomp} % provide euro and other symbols
\else % if luatex or xetex
  \usepackage{unicode-math}
  \defaultfontfeatures{Scale=MatchLowercase}
  \defaultfontfeatures[\rmfamily]{Ligatures=TeX,Scale=1}
\fi
% Use upquote if available, for straight quotes in verbatim environments
\IfFileExists{upquote.sty}{\usepackage{upquote}}{}
\IfFileExists{microtype.sty}{% use microtype if available
  \usepackage[]{microtype}
  \UseMicrotypeSet[protrusion]{basicmath} % disable protrusion for tt fonts
}{}
\makeatletter
\@ifundefined{KOMAClassName}{% if non-KOMA class
  \IfFileExists{parskip.sty}{%
    \usepackage{parskip}
  }{% else
    \setlength{\parindent}{0pt}
    \setlength{\parskip}{6pt plus 2pt minus 1pt}}
}{% if KOMA class
  \KOMAoptions{parskip=half}}
\makeatother
\usepackage{xcolor}
\IfFileExists{xurl.sty}{\usepackage{xurl}}{} % add URL line breaks if available
\IfFileExists{bookmark.sty}{\usepackage{bookmark}}{\usepackage{hyperref}}
\hypersetup{
  pdftitle={mapping covid cases {[}updates daily{]}},
  hidelinks,
  pdfcreator={LaTeX via pandoc}}
\urlstyle{same} % disable monospaced font for URLs
\usepackage[margin=1in]{geometry}
\usepackage{graphicx}
\makeatletter
\def\maxwidth{\ifdim\Gin@nat@width>\linewidth\linewidth\else\Gin@nat@width\fi}
\def\maxheight{\ifdim\Gin@nat@height>\textheight\textheight\else\Gin@nat@height\fi}
\makeatother
% Scale images if necessary, so that they will not overflow the page
% margins by default, and it is still possible to overwrite the defaults
% using explicit options in \includegraphics[width, height, ...]{}
\setkeys{Gin}{width=\maxwidth,height=\maxheight,keepaspectratio}
% Set default figure placement to htbp
\makeatletter
\def\fps@figure{htbp}
\makeatother
\setlength{\emergencystretch}{3em} % prevent overfull lines
\providecommand{\tightlist}{%
  \setlength{\itemsep}{0pt}\setlength{\parskip}{0pt}}
\setcounter{secnumdepth}{-\maxdimen} % remove section numbering
\ifluatex
  \usepackage{selnolig}  % disable illegal ligatures
\fi

\title{mapping covid cases {[}updates daily{]}}
\author{}
\date{\vspace{-2.5em}december 2, 2021}

\begin{document}
\maketitle

\hypertarget{q2-introduction}{%
\section{Q2: Introduction}\label{q2-introduction}}

Our analysis focuses on the confirmed COVID-19 cases in seven of the
southern states, taken from NYTimes dataset\footnote{NYTimes, (2021).
  Covid-19 Data {[}Data set{]}}. These seven states are key states
because according to Hematology News\footnote{Publish date: May 12,
  2020, \& Franki, R. (2021, August 26). States vary in vulnerability to
  covid-19 impact. MDedge Hematology and Oncology. Retrieved November
  20, 2021, from
  \url{https://www.mdedge.com/hematology-oncology/article/222125/coronavirus-updates/states-vary-vulnerability-covid-19-impact}.},
these states are among the top 10 states that have the most vulnerable
population. The dataframe we used updates daily, and thus the data
continually changes. As can be seen, most of Florida has the darker
shade of red, meaning it has significantly more cases of the virus for
the current date the data is showing. Furthermore, most of the cases in
Florida appear to be in the counties towards the bottom of the state,
which could be caused by the major cities, such as Orlando, Miami, and
Tampa,\footnote{jochenf36, (2014). US State Capitals {[}Data set{]}}
being located there.

The right side of the map also appear to have heavier amounts of the
darker hues, meaning the states of South Carolina, Georgia, Florida and
Alabama have more confirmed cases. Another observation is that the
darker shades appear adjacent to each other, which can be attributed to
the virus's tendency to be extremely contagious. Being so close
together, can cause more exposure and positive cases. Lastly, the west
of Texas have fewer cases compared to the rest of the Southern region of
the country, which may be due to the fact that those sections are
occupied with significantly less population\footnote{Texas population
  2021. Texas Population 2021 (Demographics, Maps, Graphs). (n.d.).
  Retrieved November 20, 2021, from
  \url{https://worldpopulationreview.com/states/texas-population}.}.
Having less of a population, means less spread of the virus. Overall, by
looking at the data we conclude that the heavily densely populated areas
in the Southern Region of the United States are significantly being more
affected by this virus, compared to the less populated areas.

\hypertarget{q3-the-map}{%
\section{Q3: The map}\label{q3-the-map}}

\includegraphics{mapping-covid-cases-{[}live{]}_files/figure-latex/unnamed-chunk-4-1.pdf}

\hypertarget{additional-analysis}{%
\section{Additional Analysis}\label{additional-analysis}}

We used a shapefile to outline the boundaries of the seven specific
states we decided to focus on\footnote{Bureau, U. S. C. (2021, October
  8). Cartographic boundary files - shapefile. Census.gov. Retrieved
  November 25, 2021, from
  \url{https://www.census.gov/geographies/mapping-files/time-series/geo/carto-boundary-file.html}.}.

\end{document}
